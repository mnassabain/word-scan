\documentclass[11pt]{article}

% \usepackage[francais]{babel}
\usepackage[utf8]{inputenc}
\usepackage[margin=1in]{geometry}


\title{Analyse théorique du projet 2018}

\author{
    CHRISTOFFEL Quentin
    \and
    NASSABAIN Marco
}
\date{Lundi, 7 Mai 2018}

\begin{document}

\maketitle


\section{Introduction}
    Ce rapport est une analyse théorique du programme donné pour le projet 2018.
    Le programme doit prendre en entrée un fichier. Il lit le fichier ligne
    par ligne et il insère les mots trouvés et leurs occurences dans un arbre
    binaire de recherche équilibré.\\

    Le programme est testé sur \texttt{Turing} et compile sans problèmes. Il n'a
    pas d'erreur de mémoire. Le
    programme ne traite pas les sauts de ligne comme fin de phrase. En effet,
    il est capable de différencier les mots des séparateurs et donc traiter un
    point comme la fin d'une phrase. Le programme gère aussi les lettres en
    majuscules et les accents. Vous trouverez plus de détails dans les fichiers
    source. Le Makefile comporte un jeu de tests pour tester le programme.\\


\section{Manuel d'utilisation}
    Pour lancer le programme il faut appeler le fichier executable avec un
    argument: le fichier depuis vous voulez lire le texte.\\
        \texttt{./analyser fichier}\\

    \noindent
    Voici une liste des options que vous pouvez utiliser:

    \begin{itemize}
        \item \texttt{-a}: Affiche une aide pour l'utilisation.
        \item \texttt{-h}: Affiche la hauteur de l'arbre binaire de recherche.
        \item \texttt{-p}: Affiche la profondeur moyenne des noeuds de l'arbre.
        \item \texttt{-n}: Affiche le nombre de noeuds de l'arbre.
        \item \texttt{-e}: Teste si l'arbre est bien équilibré.
        \item \texttt{-u}: Passage en mode utf-8 pour le traîtement des accents.
        \item \texttt{-i}: Lance le programme en mode intéractif.
        \item \texttt{-t}: Lance un jeu de tests. (Peut être lancé avec -i pour 
        faire un jeu de testes sur le mode intéractif)
    \end{itemize}

    Le programme va lire depuis le fichier et afficher l'arbre binaire équilibré
    de recherche associé au texte.



\section{Ensemble ordonné}

    \subsection{Implémentation}
    Nous avons choisi de représenter
    la structure de ensemble ordonné par une table.
    Cette table contient un tableau, qui stocke les
    élements triés par ordre croissant. La structure contient aussi
    la capacité du tableau ainsi que le nombre d'élements dans le tableau.
    Il est important de remarquer que la capacité est le
    nombre maximum d'éléments qu'il peut stocker
    avant de faire une réallocation de mémoire, s'il n'y a plus assez
    de place dans le tableau.

    \begin{verbatim}
    typedef struct s_set
    {
        int * elements;
        int max_elt;
        int n_elt;

    } * OrderedSet;
    \end{verbatim}

    Nous avons décidé de représenter cette structure sous cette forme
    car l'insertion nécessite de faire une recherche
    pour voir si l'ensemble contient déjà l'élément, ce qui est plus
    efficace avec une recherche dichotomique dans un tableau trié.

    \subsection{Fonctionnement de la fonction intersection}
    Au debut de l'algorithme nous choisisons l'ensemble qui contient moins
    d'éléments que l'autre. Nous faisons un parcours linéaire sur celui-ci
    et nous testons si les éléments sont présents dans l'autre ensemble.
    Nous créeons un nouvel ensemble qui contient l'intersection.

    \subsection{Complexité}

        \subsubsection{Insertion dans l'ensemble}
        Pour insérer un élément dans l'ensemble nous faisons d'abord une
        recherche dichotomique pour voir si l'élément est présent, si
        pendant cette recherche on trouve l'élement alors on s'arrête et
        on retourne l'ensemble d'entrée. Si l'élément n'est pas présent,
        on effectue un décalage des éléments du tableau vers la droite,
        à partir de la position trouvée. Si ce décalage entraîne un dépassement
        de la capacité du tableau on effectue une réallocation.
        Ensuite on insère l'élement à cette position.\\

        La complexité de la recherche dichotomique est en $\Theta(\log{}n)$ pour
        un ensemble de n éléments. Le décalage du tableau a une complexité en
        $\Theta(n)$ dans le pire cas. Ce cas correspond à une insertion
        en première position dans l'ensemble ordonné.

        \subsubsection{Appartenance d'un élément}
        Quand il faut tester si un élément appartient à un ensemble on fait
        une recherche dichotomique. Si on trouve l'élément à la position donnée
        par la recherche, on renvoie \texttt{true} et \texttt{false} sinon.\\

        La complexité de cet algorithme est $\Theta(\log{}n)$. Le pire cas
        est quand l'élément est en première ou dernière position.

        \subsubsection{Intersection de 2 ensembles}
        Pour réaliser l'intersection de deux ensembles nous faisons un parcours
        linéaire de l'ensemble qui contient moins d'éléments que l'autre. Ensuite
        on teste si cet élément appartient au deuxième ensemble avec une
        recherche dichotomique.\\

        La complexité de cet algorithme est $\Theta(\log{}m)$.


\section{Arbre binaire de recherche}

    \subsection{Choix d'implémentations pour l'arbre binaire de recherche}
    Pour représenter un arbre binaire de recherche nous avons choisi de créer
    une structure qui contient les données (le mot et ses occurences), les fils
    du noeud et le facteur d'équilibrage du noeud.

    \begin{verbatim}
    typedef struct s_arbre
    {
        char * mot;
        OrderedSet positions;

        int eq;

        struct s_arbre *fg, *fd;

    } Noeud, *SearchTree;
    \end{verbatim}

    On a choisi cette structure car elle permet de simplifier les opérations
    sur les arbres. On stocke le facteur d'équilibrage \texttt{eq} dans cette
    structure pour améliorer la complexité lors de l'équilibrage de l'arbre.

    \subsection{Fonctionnement des fonctions getAverageDepth et isBalanced}
    La fonction \texttt{getAverageDepth()} utilise la fonction \texttt{lci()}
    qui permet de calculer la longueur de cheminement interne et
    divise cette valeur par le nombre de noeuds de l'arbre obtenus
    grâce à la fonction \texttt{getNumberString()}. Ce calcul nous donne la
    hauteur moyenne de tous les noeuds dans l'arbre.\\

    La fonction \texttt{isBalanced()} commence par la racine. Si la racine est
    déséquilibrée elle renvoie \texttt{false}. Si elle est équilibrée, on
    appelle cette fonction sur le fils gauche. Si le fils gauche est déséquilibré, on
    renvoie \texttt{false} et sinon on appelle la fonction sur le fils droit.
    Si lui aussi est équilibré, on renvoie \texttt{true} et sinon \texttt{false}.
    Or cette fonction est récursive donc on va pouvoir répéter ces opérations pour
    chaque noeud dans l'arbre pour tester son équilibrage.

    \subsection{Complexité de FindCooccurrences}

        \subsubsection{Hauteur moyenne d'un noeud}
        Prenons l'exemple de deux arbres avec le même nombre de noeuds dont un
        est équilibré et l'autre non.
        Le facteur d'équilibrage d'un noeud correspond à la difference des
        hauteurs de ses fils. On peut remarquer que cette difference va être plus
        grande si l'arbre est déséquilibré, car rien n'oblige l'arbre à avoir
        le même nombre de noeuds de chaque côté. Cependant, dans un arbre
        équilibré, cette difference est minimale. L'arbre impose que la difference
        en valeur absolue entre les hauteurs du fils gauche et du fils droit soit
        nulle ou un. Donc un arbre non équilibré aura plus de noeuds avec une
        hauteur importante. Cela résulte en une hauteur moyenne
        supérieure à la hauteur moyenne d'un arbre équilibré.\\

        Cela peut avoir un grand impact sur les performances d'un programme.
        Lors du parcours de l'arbre non équilibré il va falloir parcourir beaucoup
        plus de noeuds pour trouver ce qu'on cherche. Une recherche dans un arbre
        équilibré est très similaire à une recherche dichotomique dans un ensemble.


        \subsubsection{Complexité dans le pire cas dans un arbre non équilibré}
        La première phase de cet algorithme est de trouver les mots donnés dans
        l'arbre binaire. Dans le cas d'un arbre non équilibré, le parcours va être
        plus lent. Effectivement, dans le pire cas, les mot recherchés sont des
        feuilles ayant une profondeur égale à la hauteur de l'arbre. Pour les
        trouver il va falloir parcourir l'arbre deux fois. La complexité de
        cette recherche est de $\Theta(n)$, où n est le nombre de noeuds dans
        l'arbre.\\

        Après avoir fait cela on fait appel à la fonction \texttt{intersect()}
        sur les ensembles des occurences des mots cherchés, dont on a déjà
        calculé la complexité, qui est donc de $\Theta(k\log{}k)$ dans le pire cas,
        k étant le nombre maximal de phrases dans lesquelles un mot apparaît.

        \subsubsection{Complexité dans le pire cas dans un arbre équilibré}
        Dans le cas d'un arbre équilibré, la recherche est beaucoup plus rapide.
        En effet, dans un arbre de \texttt{n} éléments, dans le pire cas il va falloir
        parcourir ${\log_2(n+1)}$ noeuds. Donc la complexité de la recherche
        est $\Theta(\log{}n)$ et pas de $\Theta(n)$ comme dans le cas d'un arbre
        binaire non équilibré.

        La complexité de l'intersection est la même.


    \section{Exemple}

    \begin{verbatim}
    $ ./analyser toto
       /--waldo: [ 3 ]
       |
    /--qux: [ 2 ]
    |  |
    |  \--grault: [ 1, 3 ]
    |
    foo: [ 1, 2, 3 ]
    |
    |  /--corge: [ 2, 3 ]
    |  |
    \--baz: [ 1 ]
       |
       \--bar: [ 1, 2 ]
    \end{verbatim}


\end{document}
